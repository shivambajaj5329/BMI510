% Options for packages loaded elsewhere
\PassOptionsToPackage{unicode}{hyperref}
\PassOptionsToPackage{hyphens}{url}
%
\documentclass[
]{article}
\usepackage{amsmath,amssymb}
\usepackage{lmodern}
\usepackage{iftex}
\ifPDFTeX
  \usepackage[T1]{fontenc}
  \usepackage[utf8]{inputenc}
  \usepackage{textcomp} % provide euro and other symbols
\else % if luatex or xetex
  \usepackage{unicode-math}
  \defaultfontfeatures{Scale=MatchLowercase}
  \defaultfontfeatures[\rmfamily]{Ligatures=TeX,Scale=1}
\fi
% Use upquote if available, for straight quotes in verbatim environments
\IfFileExists{upquote.sty}{\usepackage{upquote}}{}
\IfFileExists{microtype.sty}{% use microtype if available
  \usepackage[]{microtype}
  \UseMicrotypeSet[protrusion]{basicmath} % disable protrusion for tt fonts
}{}
\makeatletter
\@ifundefined{KOMAClassName}{% if non-KOMA class
  \IfFileExists{parskip.sty}{%
    \usepackage{parskip}
  }{% else
    \setlength{\parindent}{0pt}
    \setlength{\parskip}{6pt plus 2pt minus 1pt}}
}{% if KOMA class
  \KOMAoptions{parskip=half}}
\makeatother
\usepackage{xcolor}
\usepackage[margin=1in]{geometry}
\usepackage{color}
\usepackage{fancyvrb}
\newcommand{\VerbBar}{|}
\newcommand{\VERB}{\Verb[commandchars=\\\{\}]}
\DefineVerbatimEnvironment{Highlighting}{Verbatim}{commandchars=\\\{\}}
% Add ',fontsize=\small' for more characters per line
\usepackage{framed}
\definecolor{shadecolor}{RGB}{248,248,248}
\newenvironment{Shaded}{\begin{snugshade}}{\end{snugshade}}
\newcommand{\AlertTok}[1]{\textcolor[rgb]{0.94,0.16,0.16}{#1}}
\newcommand{\AnnotationTok}[1]{\textcolor[rgb]{0.56,0.35,0.01}{\textbf{\textit{#1}}}}
\newcommand{\AttributeTok}[1]{\textcolor[rgb]{0.77,0.63,0.00}{#1}}
\newcommand{\BaseNTok}[1]{\textcolor[rgb]{0.00,0.00,0.81}{#1}}
\newcommand{\BuiltInTok}[1]{#1}
\newcommand{\CharTok}[1]{\textcolor[rgb]{0.31,0.60,0.02}{#1}}
\newcommand{\CommentTok}[1]{\textcolor[rgb]{0.56,0.35,0.01}{\textit{#1}}}
\newcommand{\CommentVarTok}[1]{\textcolor[rgb]{0.56,0.35,0.01}{\textbf{\textit{#1}}}}
\newcommand{\ConstantTok}[1]{\textcolor[rgb]{0.00,0.00,0.00}{#1}}
\newcommand{\ControlFlowTok}[1]{\textcolor[rgb]{0.13,0.29,0.53}{\textbf{#1}}}
\newcommand{\DataTypeTok}[1]{\textcolor[rgb]{0.13,0.29,0.53}{#1}}
\newcommand{\DecValTok}[1]{\textcolor[rgb]{0.00,0.00,0.81}{#1}}
\newcommand{\DocumentationTok}[1]{\textcolor[rgb]{0.56,0.35,0.01}{\textbf{\textit{#1}}}}
\newcommand{\ErrorTok}[1]{\textcolor[rgb]{0.64,0.00,0.00}{\textbf{#1}}}
\newcommand{\ExtensionTok}[1]{#1}
\newcommand{\FloatTok}[1]{\textcolor[rgb]{0.00,0.00,0.81}{#1}}
\newcommand{\FunctionTok}[1]{\textcolor[rgb]{0.00,0.00,0.00}{#1}}
\newcommand{\ImportTok}[1]{#1}
\newcommand{\InformationTok}[1]{\textcolor[rgb]{0.56,0.35,0.01}{\textbf{\textit{#1}}}}
\newcommand{\KeywordTok}[1]{\textcolor[rgb]{0.13,0.29,0.53}{\textbf{#1}}}
\newcommand{\NormalTok}[1]{#1}
\newcommand{\OperatorTok}[1]{\textcolor[rgb]{0.81,0.36,0.00}{\textbf{#1}}}
\newcommand{\OtherTok}[1]{\textcolor[rgb]{0.56,0.35,0.01}{#1}}
\newcommand{\PreprocessorTok}[1]{\textcolor[rgb]{0.56,0.35,0.01}{\textit{#1}}}
\newcommand{\RegionMarkerTok}[1]{#1}
\newcommand{\SpecialCharTok}[1]{\textcolor[rgb]{0.00,0.00,0.00}{#1}}
\newcommand{\SpecialStringTok}[1]{\textcolor[rgb]{0.31,0.60,0.02}{#1}}
\newcommand{\StringTok}[1]{\textcolor[rgb]{0.31,0.60,0.02}{#1}}
\newcommand{\VariableTok}[1]{\textcolor[rgb]{0.00,0.00,0.00}{#1}}
\newcommand{\VerbatimStringTok}[1]{\textcolor[rgb]{0.31,0.60,0.02}{#1}}
\newcommand{\WarningTok}[1]{\textcolor[rgb]{0.56,0.35,0.01}{\textbf{\textit{#1}}}}
\usepackage{longtable,booktabs,array}
\usepackage{calc} % for calculating minipage widths
% Correct order of tables after \paragraph or \subparagraph
\usepackage{etoolbox}
\makeatletter
\patchcmd\longtable{\par}{\if@noskipsec\mbox{}\fi\par}{}{}
\makeatother
% Allow footnotes in longtable head/foot
\IfFileExists{footnotehyper.sty}{\usepackage{footnotehyper}}{\usepackage{footnote}}
\makesavenoteenv{longtable}
\usepackage{graphicx}
\makeatletter
\def\maxwidth{\ifdim\Gin@nat@width>\linewidth\linewidth\else\Gin@nat@width\fi}
\def\maxheight{\ifdim\Gin@nat@height>\textheight\textheight\else\Gin@nat@height\fi}
\makeatother
% Scale images if necessary, so that they will not overflow the page
% margins by default, and it is still possible to overwrite the defaults
% using explicit options in \includegraphics[width, height, ...]{}
\setkeys{Gin}{width=\maxwidth,height=\maxheight,keepaspectratio}
% Set default figure placement to htbp
\makeatletter
\def\fps@figure{htbp}
\makeatother
\setlength{\emergencystretch}{3em} % prevent overfull lines
\providecommand{\tightlist}{%
  \setlength{\itemsep}{0pt}\setlength{\parskip}{0pt}}
\setcounter{secnumdepth}{-\maxdimen} % remove section numbering
\ifLuaTeX
  \usepackage{selnolig}  % disable illegal ligatures
\fi
\IfFileExists{bookmark.sty}{\usepackage{bookmark}}{\usepackage{hyperref}}
\IfFileExists{xurl.sty}{\usepackage{xurl}}{} % add URL line breaks if available
\urlstyle{same} % disable monospaced font for URLs
\hypersetup{
  pdftitle={Homework 4},
  pdfauthor={Shivam Bajaj},
  hidelinks,
  pdfcreator={LaTeX via pandoc}}

\title{Homework 4}
\author{Shivam Bajaj}
\date{February 6, 2023}

\begin{document}
\maketitle

\begin{center}\rule{0.5\linewidth}{0.5pt}\end{center}

\begin{enumerate}
\def\labelenumi{\arabic{enumi}.}
\tightlist
\item
  There is a dataset of clinical variables and self-reported fall
  history from people with Parkinson's disease available at
  {[}\url{https://jlucasmckay.bmi.emory.edu/global/mckay_2021/S1.csv}{]}.
  Identify the sample mean and sample standard deviation of the
  \texttt{Age} variable. Exclude any cases under 35 years old. \textbf{1
  point}
\end{enumerate}

\begin{Shaded}
\begin{Highlighting}[]

\NormalTok{data }\OtherTok{\textless{}{-}} \FunctionTok{read.csv}\NormalTok{(}\StringTok{"https://jlucasmckay.bmi.emory.edu/global/mckay\_2021/S1.csv"}\NormalTok{)}
\NormalTok{data }\OtherTok{\textless{}{-}}\NormalTok{ data[data}\SpecialCharTok{$}\NormalTok{Age }\SpecialCharTok{\textgreater{}=} \DecValTok{35}\NormalTok{,]}

\NormalTok{m }\OtherTok{\textless{}{-}} \FunctionTok{mean}\NormalTok{(data}\SpecialCharTok{$}\NormalTok{Age)}
\NormalTok{s }\OtherTok{\textless{}{-}} \FunctionTok{sd}\NormalTok{(data}\SpecialCharTok{$}\NormalTok{Age)}
\NormalTok{l }\OtherTok{\textless{}{-}} \FunctionTok{length}\NormalTok{(data}\SpecialCharTok{$}\NormalTok{Age)}
\FunctionTok{print}\NormalTok{(m)}
\DocumentationTok{\#\# [1] 67.36935}

\FunctionTok{print}\NormalTok{(s)}
\DocumentationTok{\#\# [1] 7.418488}
\end{Highlighting}
\end{Shaded}

\begin{center}\rule{0.5\linewidth}{0.5pt}\end{center}

\begin{enumerate}
\def\labelenumi{\arabic{enumi}.}
\setcounter{enumi}{1}
\tightlist
\item
  Calculate the likelihood of the estimates for the mean and standard
  deviation you have obtained given the sample. (Assume a normal
  distribution.) as you sweep the estimate of average age through ±2
  years. Specifically, create a function \texttt{Lik(a)} that returns
  the likelihood of the data given an estimate of the population mean
  age \texttt{a} \textbf{(1/2 point)} and plot the likelihood as a
  function of \texttt{a} as you sweep it over ±2 years. \textbf{(1/2
  point)}
\end{enumerate}

\begin{Shaded}
\begin{Highlighting}[]
\CommentTok{\# Define the likelihood function}

\NormalTok{Age }\OtherTok{\textless{}{-}}\NormalTok{ data}\SpecialCharTok{$}\NormalTok{Age}
\NormalTok{Lik }\OtherTok{\textless{}{-}} \ControlFlowTok{function}\NormalTok{(a) \{}

\NormalTok{  likelihood }\OtherTok{\textless{}{-}} \FunctionTok{prod}\NormalTok{(}\FunctionTok{dnorm}\NormalTok{(Age, }\AttributeTok{mean =}\NormalTok{ a, }\AttributeTok{sd =}\NormalTok{ s))}
  \FunctionTok{return}\NormalTok{(likelihood)}
\NormalTok{\}}


\FunctionTok{library}\NormalTok{(ggplot2)}

\NormalTok{a\_range }\OtherTok{\textless{}{-}} \FunctionTok{seq}\NormalTok{(}\AttributeTok{from =}\NormalTok{ m }\SpecialCharTok{{-}} \DecValTok{2}\NormalTok{, }\AttributeTok{to =}\NormalTok{ m }\SpecialCharTok{+} \DecValTok{2}\NormalTok{, }\AttributeTok{by =} \FloatTok{0.01}\NormalTok{)}
\NormalTok{likelihood\_values }\OtherTok{\textless{}{-}} \FunctionTok{sapply}\NormalTok{(a\_range, Lik)}


\FunctionTok{ggplot}\NormalTok{(}\FunctionTok{data.frame}\NormalTok{(}\AttributeTok{a =}\NormalTok{ a\_range, }\AttributeTok{likelihood =}\NormalTok{ likelihood\_values), }\FunctionTok{aes}\NormalTok{(}\AttributeTok{x =}\NormalTok{ a, }\AttributeTok{y =}\NormalTok{ likelihood)) }\SpecialCharTok{+}
  \FunctionTok{geom\_line}\NormalTok{() }\SpecialCharTok{+}
  \FunctionTok{ggtitle}\NormalTok{(}\StringTok{"Likelihood as a function of the estimate of the population mean age"}\NormalTok{) }\SpecialCharTok{+}
  \FunctionTok{xlab}\NormalTok{(}\StringTok{"Estimate of the population mean age (a)"}\NormalTok{) }\SpecialCharTok{+}
  \FunctionTok{ylab}\NormalTok{(}\StringTok{"Likelihood"}\NormalTok{)}
\end{Highlighting}
\end{Shaded}

\includegraphics{HW-04_files/figure-latex/unnamed-chunk-2-1.pdf}

\begin{center}\rule{0.5\linewidth}{0.5pt}\end{center}

\begin{enumerate}
\def\labelenumi{\arabic{enumi}.}
\setcounter{enumi}{2}
\tightlist
\item
  Now plot the \emph{log} likelihood as you sweep the estimate of
  average age through ±2 years. \textbf{(1 point)} \emph{(Note that you
  have to use a sum, not a product, here.)}
\end{enumerate}

\begin{Shaded}
\begin{Highlighting}[]

\NormalTok{Age }\OtherTok{\textless{}{-}}\NormalTok{ data}\SpecialCharTok{$}\NormalTok{Age}
\NormalTok{logLik }\OtherTok{\textless{}{-}} \ControlFlowTok{function}\NormalTok{(a) \{}

\NormalTok{  likelihood }\OtherTok{\textless{}{-}} \FunctionTok{sum}\NormalTok{(}\FunctionTok{dnorm}\NormalTok{(Age, }\AttributeTok{mean =}\NormalTok{ a, }\AttributeTok{sd =}\NormalTok{ s, }\AttributeTok{log =} \ConstantTok{TRUE}\NormalTok{))}
  \FunctionTok{return}\NormalTok{(likelihood)}
\NormalTok{\}}


\FunctionTok{library}\NormalTok{(ggplot2)}

\NormalTok{a\_range }\OtherTok{\textless{}{-}} \FunctionTok{seq}\NormalTok{(}\AttributeTok{from =}\NormalTok{ m }\SpecialCharTok{{-}} \DecValTok{2}\NormalTok{, }\AttributeTok{to =}\NormalTok{ m }\SpecialCharTok{+} \DecValTok{2}\NormalTok{, }\AttributeTok{by =} \FloatTok{0.01}\NormalTok{)}
\NormalTok{log\_likelihood\_values }\OtherTok{\textless{}{-}} \FunctionTok{sapply}\NormalTok{(a\_range, logLik)}


\FunctionTok{ggplot}\NormalTok{(}\FunctionTok{data.frame}\NormalTok{(}\AttributeTok{a =}\NormalTok{ a\_range, }\AttributeTok{log\_likelihood =}\NormalTok{ log\_likelihood\_values), }\FunctionTok{aes}\NormalTok{(}\AttributeTok{x =}\NormalTok{ a, }\AttributeTok{y =}\NormalTok{ log\_likelihood)) }\SpecialCharTok{+}
  \FunctionTok{geom\_line}\NormalTok{() }\SpecialCharTok{+}
  \FunctionTok{ggtitle}\NormalTok{(}\StringTok{"Likelihood as a function of the estimate of the population mean age"}\NormalTok{) }\SpecialCharTok{+}
  \FunctionTok{xlab}\NormalTok{(}\StringTok{"Estimate of the population mean age (a)"}\NormalTok{) }\SpecialCharTok{+}
  \FunctionTok{ylab}\NormalTok{(}\StringTok{"Log Likelihood"}\NormalTok{)}
\end{Highlighting}
\end{Shaded}

\includegraphics{HW-04_files/figure-latex/unnamed-chunk-3-1.pdf}

\begin{center}\rule{0.5\linewidth}{0.5pt}\end{center}

\begin{enumerate}
\def\labelenumi{\arabic{enumi}.}
\setcounter{enumi}{3}
\tightlist
\item
  Using the answers for the last two questions, use a grid-based
  approach to find the maximum likelihood estimator for \texttt{a}.
  \textbf{(1/2 point)} Do the same for the maximum log-likelihood
  estimator. \textbf{(1/2 point)} Do they differ?
\end{enumerate}

\begin{Shaded}
\begin{Highlighting}[]

\NormalTok{grid }\OtherTok{\textless{}{-}} \FunctionTok{seq}\NormalTok{(}\AttributeTok{from =} \FunctionTok{min}\NormalTok{(Age), }\AttributeTok{to =} \FunctionTok{max}\NormalTok{(Age), }\AttributeTok{by =} \DecValTok{1}\NormalTok{)}

\CommentTok{\#assigning a empty vector of length grid}
\NormalTok{lik\_values }\OtherTok{\textless{}{-}} \FunctionTok{numeric}\NormalTok{(}\FunctionTok{length}\NormalTok{(grid))}
\NormalTok{loglik\_values }\OtherTok{\textless{}{-}} \FunctionTok{numeric}\NormalTok{(}\FunctionTok{length}\NormalTok{(grid))}

\CommentTok{\#calculating log likelihood and likelihood for all values in grid}
\ControlFlowTok{for}\NormalTok{ (i }\ControlFlowTok{in} \DecValTok{1}\SpecialCharTok{:}\FunctionTok{length}\NormalTok{(grid)) \{}
\NormalTok{  lik\_values[i] }\OtherTok{\textless{}{-}} \FunctionTok{Lik}\NormalTok{(grid[i])}
\NormalTok{  loglik\_values[i] }\OtherTok{\textless{}{-}} \FunctionTok{logLik}\NormalTok{(grid[i])}
\NormalTok{\}}

\CommentTok{\#calculating max likelihood and log likelihood}
\NormalTok{max\_index\_likelihood }\OtherTok{\textless{}{-}} \FunctionTok{which.max}\NormalTok{(lik\_values)}
\NormalTok{max\_index\_loglikelihood }\OtherTok{\textless{}{-}} \FunctionTok{which.max}\NormalTok{(loglik\_values)}

\NormalTok{a\_hat\_likelihood }\OtherTok{\textless{}{-}}\NormalTok{ grid[max\_index\_likelihood]}
\NormalTok{a\_hat\_loglikelihood }\OtherTok{\textless{}{-}}\NormalTok{ grid[max\_index\_loglikelihood]}


\FunctionTok{cat}\NormalTok{(}\StringTok{"liklihood age"}\NormalTok{,a\_hat\_likelihood)}
\DocumentationTok{\#\# liklihood age 67}
\FunctionTok{cat}\NormalTok{(}\StringTok{"}\SpecialCharTok{\textbackslash{}n}\StringTok{log liklihood age"}\NormalTok{, a\_hat\_loglikelihood)}
\DocumentationTok{\#\# }
\DocumentationTok{\#\# log liklihood age 67}

\CommentTok{\#we see that they do not differ.}
\end{Highlighting}
\end{Shaded}

\begin{center}\rule{0.5\linewidth}{0.5pt}\end{center}

\begin{enumerate}
\def\labelenumi{\arabic{enumi}.}
\setcounter{enumi}{4}
\tightlist
\item
  One of the things we noted was that using the sample standard
  deviation to estimate the population standard deviation can bias the
  estimate. Therefore, we often see \texttt{N-1} normalization in the
  standard deviation equation instead of \texttt{N}. Calculate and
  compare the likelihood of the data under the biased and unbiased
  estimators for the standard deviation. Which estimate for the standard
  deviation is larger? Which estimate is more likely? \textbf{(1 point)}
\end{enumerate}

\begin{Shaded}
\begin{Highlighting}[]


\NormalTok{bias\_sd }\OtherTok{\textless{}{-}} \FunctionTok{sqrt}\NormalTok{(}\FunctionTok{sum}\NormalTok{((data}\SpecialCharTok{$}\NormalTok{Age }\SpecialCharTok{{-}} \FunctionTok{mean}\NormalTok{(data}\SpecialCharTok{$}\NormalTok{Age))}\SpecialCharTok{\^{}}\DecValTok{2}\NormalTok{) }\SpecialCharTok{/} \FunctionTok{length}\NormalTok{(data}\SpecialCharTok{$}\NormalTok{Age))}
\NormalTok{unbias\_sd }\OtherTok{\textless{}{-}} \FunctionTok{sqrt}\NormalTok{(}\FunctionTok{sum}\NormalTok{((data}\SpecialCharTok{$}\NormalTok{Age }\SpecialCharTok{{-}} \FunctionTok{mean}\NormalTok{(data}\SpecialCharTok{$}\NormalTok{Age))}\SpecialCharTok{\^{}}\DecValTok{2}\NormalTok{) }\SpecialCharTok{/}\NormalTok{ (}\FunctionTok{length}\NormalTok{(data}\SpecialCharTok{$}\NormalTok{Age) }\SpecialCharTok{{-}} \DecValTok{1}\NormalTok{))}


\NormalTok{Lik\_biased }\OtherTok{\textless{}{-}} \ControlFlowTok{function}\NormalTok{(a) \{}
\NormalTok{  likelihood }\OtherTok{\textless{}{-}} \FunctionTok{prod}\NormalTok{(}\FunctionTok{dnorm}\NormalTok{(Age, }\AttributeTok{mean =}\NormalTok{ a, }\AttributeTok{sd =}\NormalTok{ bias\_sd))}\CommentTok{\#sd(N)}
  \FunctionTok{return}\NormalTok{(likelihood)}
\NormalTok{\}}


\NormalTok{Lik\_unbiased }\OtherTok{\textless{}{-}} \ControlFlowTok{function}\NormalTok{(a) \{}
\NormalTok{  likelihood }\OtherTok{\textless{}{-}} \FunctionTok{prod}\NormalTok{(}\FunctionTok{dnorm}\NormalTok{(Age, }\AttributeTok{mean =}\NormalTok{ a, }\AttributeTok{sd =}\NormalTok{ unbias\_sd))}\CommentTok{\#sd(N{-}1)}
  \FunctionTok{return}\NormalTok{(likelihood)}
\NormalTok{\}}


\NormalTok{a\_range }\OtherTok{\textless{}{-}} \FunctionTok{seq}\NormalTok{(}\AttributeTok{from =}\NormalTok{ m }\SpecialCharTok{{-}} \DecValTok{2}\NormalTok{, }\AttributeTok{to =}\NormalTok{ m }\SpecialCharTok{+} \DecValTok{2}\NormalTok{, }\AttributeTok{by =} \FloatTok{0.01}\NormalTok{)}
\NormalTok{likelihood\_values }\OtherTok{\textless{}{-}} \FunctionTok{sapply}\NormalTok{(a\_range, Lik\_unbiased)}


\FunctionTok{ggplot}\NormalTok{(}\FunctionTok{data.frame}\NormalTok{(}\AttributeTok{a =}\NormalTok{ a\_range, }\AttributeTok{likelihood =}\NormalTok{ likelihood\_values), }\FunctionTok{aes}\NormalTok{(}\AttributeTok{x =}\NormalTok{ a, }\AttributeTok{y =}\NormalTok{ likelihood)) }\SpecialCharTok{+}
  \FunctionTok{geom\_line}\NormalTok{() }\SpecialCharTok{+}
  \FunctionTok{ggtitle}\NormalTok{(}\StringTok{"Likelihood estimate of the population mean age unbiased"}\NormalTok{) }\SpecialCharTok{+}
  \FunctionTok{xlab}\NormalTok{(}\StringTok{"Estimate of the population mean age (a)"}\NormalTok{) }\SpecialCharTok{+}
  \FunctionTok{ylab}\NormalTok{(}\StringTok{"Log Likelihood"}\NormalTok{)}
\end{Highlighting}
\end{Shaded}

\includegraphics{HW-04_files/figure-latex/unnamed-chunk-5-1.pdf}

\begin{Shaded}
\begin{Highlighting}[]


\NormalTok{a\_range }\OtherTok{\textless{}{-}} \FunctionTok{seq}\NormalTok{(}\AttributeTok{from =}\NormalTok{ m }\SpecialCharTok{{-}} \DecValTok{2}\NormalTok{, }\AttributeTok{to =}\NormalTok{ m }\SpecialCharTok{+} \DecValTok{2}\NormalTok{, }\AttributeTok{by =} \FloatTok{0.01}\NormalTok{)}
\NormalTok{likelihood\_values }\OtherTok{\textless{}{-}} \FunctionTok{sapply}\NormalTok{(a\_range, Lik\_biased)}


\FunctionTok{ggplot}\NormalTok{(}\FunctionTok{data.frame}\NormalTok{(}\AttributeTok{a =}\NormalTok{ a\_range, }\AttributeTok{likelihood =}\NormalTok{ likelihood\_values), }\FunctionTok{aes}\NormalTok{(}\AttributeTok{x =}\NormalTok{ a, }\AttributeTok{y =}\NormalTok{ likelihood)) }\SpecialCharTok{+}
  \FunctionTok{geom\_line}\NormalTok{() }\SpecialCharTok{+}
  \FunctionTok{ggtitle}\NormalTok{(}\StringTok{"Likelihood estimate of the population mean age biased"}\NormalTok{) }\SpecialCharTok{+}
  \FunctionTok{xlab}\NormalTok{(}\StringTok{"Estimate of the population mean age (a)"}\NormalTok{) }\SpecialCharTok{+}
  \FunctionTok{ylab}\NormalTok{(}\StringTok{"Log Likelihood"}\NormalTok{)}
\end{Highlighting}
\end{Shaded}

\includegraphics{HW-04_files/figure-latex/unnamed-chunk-5-2.pdf}

\begin{Shaded}
\begin{Highlighting}[]

\FunctionTok{cat}\NormalTok{(}\StringTok{"Biased Estimate:"}\NormalTok{, bias\_sd, }\StringTok{"}\SpecialCharTok{\textbackslash{}n}\StringTok{"}\NormalTok{)}
\DocumentationTok{\#\# Biased Estimate: 7.358418}
\FunctionTok{cat}\NormalTok{(}\StringTok{"Unbiased Estimate:"}\NormalTok{, unbias\_sd, }\StringTok{"}\SpecialCharTok{\textbackslash{}n}\StringTok{"}\NormalTok{)}
\DocumentationTok{\#\# Unbiased Estimate: 7.418488}

\CommentTok{\#Unbiased estimate is larger than biased estimate}

\CommentTok{\#Unbiased estimate is more likely as it represents real data more accurately when compared to biased estimator}
\end{Highlighting}
\end{Shaded}

\begin{center}\rule{0.5\linewidth}{0.5pt}\end{center}

\begin{enumerate}
\def\labelenumi{\arabic{enumi}.}
\setcounter{enumi}{5}
\tightlist
\item
  Let \texttt{X} be a continuous random variable with the
  piecewise-defined probability density function \(f(x)\) equal to 0.75,
  \(0≤x≤1\), 0.25, \(2≤x≤3\), 0 elsewhere. Plot the density \(f(x)\).
  \textbf{(1 point)}
\end{enumerate}

\begin{Shaded}
\begin{Highlighting}[]

\FunctionTok{library}\NormalTok{(ggplot2)}

\NormalTok{density\_function }\OtherTok{\textless{}{-}} \ControlFlowTok{function}\NormalTok{(x) \{}
  \ControlFlowTok{if}\NormalTok{ (x }\SpecialCharTok{\textgreater{}=} \DecValTok{0} \SpecialCharTok{\&}\NormalTok{ x }\SpecialCharTok{\textless{}=} \DecValTok{1}\NormalTok{) \{}
    \FunctionTok{return}\NormalTok{ (}\FloatTok{0.75}\NormalTok{)}
\NormalTok{  \} }\ControlFlowTok{else} \ControlFlowTok{if}\NormalTok{ (x }\SpecialCharTok{\textgreater{}} \DecValTok{1} \SpecialCharTok{\&}\NormalTok{ x }\SpecialCharTok{\textless{}=} \DecValTok{2}\NormalTok{) \{}
    \FunctionTok{return}\NormalTok{ (}\DecValTok{0}\NormalTok{)}
\NormalTok{  \} }\ControlFlowTok{else} \ControlFlowTok{if}\NormalTok{ (x }\SpecialCharTok{\textgreater{}} \DecValTok{2} \SpecialCharTok{\&}\NormalTok{ x }\SpecialCharTok{\textless{}=} \DecValTok{3}\NormalTok{) \{}
    \FunctionTok{return}\NormalTok{ (}\FloatTok{0.25}\NormalTok{)}
\NormalTok{  \} }\ControlFlowTok{else}\NormalTok{ \{}
    \FunctionTok{return}\NormalTok{ (}\DecValTok{0}\NormalTok{)}
\NormalTok{  \}}
\NormalTok{\}}

\NormalTok{x\_values }\OtherTok{\textless{}{-}} \FunctionTok{seq}\NormalTok{(}\AttributeTok{from =} \DecValTok{0}\NormalTok{, }\AttributeTok{to =} \DecValTok{3}\NormalTok{, }\AttributeTok{by =} \FloatTok{0.01}\NormalTok{)}
\NormalTok{density\_values }\OtherTok{\textless{}{-}} \FunctionTok{sapply}\NormalTok{(x\_values, density\_function)}
\FunctionTok{plot}\NormalTok{(x\_values, density\_values, }\AttributeTok{type =} \StringTok{"l"}\NormalTok{, }\AttributeTok{xlab =} \StringTok{"x"}\NormalTok{, }\AttributeTok{ylab =} \StringTok{"Density"}\NormalTok{, }\AttributeTok{main =} \StringTok{"Density Function"}\NormalTok{)}
\end{Highlighting}
\end{Shaded}

\includegraphics{HW-04_files/figure-latex/unnamed-chunk-6-1.pdf}

\begin{center}\rule{0.5\linewidth}{0.5pt}\end{center}

\begin{enumerate}
\def\labelenumi{\arabic{enumi}.}
\setcounter{enumi}{6}
\tightlist
\item
  Plot the cumulative density \(F(x)\). \textbf{(1 point)}
\end{enumerate}

\begin{Shaded}
\begin{Highlighting}[]

\NormalTok{cumulative\_density\_function }\OtherTok{\textless{}{-}} \ControlFlowTok{function}\NormalTok{(x) \{}
  \ControlFlowTok{if}\NormalTok{ (x }\SpecialCharTok{\textgreater{}=} \DecValTok{0} \SpecialCharTok{\&}\NormalTok{ x }\SpecialCharTok{\textless{}=} \DecValTok{1}\NormalTok{) \{}
    \FunctionTok{return}\NormalTok{ (}\FloatTok{0.75} \SpecialCharTok{*}\NormalTok{ x)}
\NormalTok{  \} }\ControlFlowTok{else} \ControlFlowTok{if}\NormalTok{ (x }\SpecialCharTok{\textgreater{}} \DecValTok{1} \SpecialCharTok{\&}\NormalTok{ x }\SpecialCharTok{\textless{}=} \DecValTok{2}\NormalTok{) \{}
    \FunctionTok{return}\NormalTok{ (}\FloatTok{0.75}\NormalTok{)}
\NormalTok{  \} }\ControlFlowTok{else} \ControlFlowTok{if}\NormalTok{ (x }\SpecialCharTok{\textgreater{}} \DecValTok{2} \SpecialCharTok{\&}\NormalTok{ x }\SpecialCharTok{\textless{}=} \DecValTok{3}\NormalTok{) \{}
    \FunctionTok{return}\NormalTok{ (}\FloatTok{0.75} \SpecialCharTok{+} \FloatTok{0.25} \SpecialCharTok{*}\NormalTok{ (x }\SpecialCharTok{{-}} \DecValTok{2}\NormalTok{))}
\NormalTok{  \} }\ControlFlowTok{else}\NormalTok{ \{}
    \FunctionTok{return}\NormalTok{ (}\DecValTok{1}\NormalTok{)}
\NormalTok{  \}}
\NormalTok{\}}

\NormalTok{x\_values }\OtherTok{\textless{}{-}} \FunctionTok{seq}\NormalTok{(}\AttributeTok{from =} \DecValTok{0}\NormalTok{, }\AttributeTok{to =} \DecValTok{3}\NormalTok{, }\AttributeTok{by =} \FloatTok{0.01}\NormalTok{)}
\NormalTok{cumulative\_density\_values }\OtherTok{\textless{}{-}} \FunctionTok{sapply}\NormalTok{(x\_values, cumulative\_density\_function)}
\FunctionTok{plot}\NormalTok{(x\_values, cumulative\_density\_values, }\AttributeTok{type =} \StringTok{"l"}\NormalTok{, }\AttributeTok{xlab =} \StringTok{"x"}\NormalTok{, }\AttributeTok{ylab =} \StringTok{"Cumulative Density"}\NormalTok{, }\AttributeTok{main =} \StringTok{"Cumulative Density Function"}\NormalTok{)}
\end{Highlighting}
\end{Shaded}

\includegraphics{HW-04_files/figure-latex/unnamed-chunk-7-1.pdf}

\begin{center}\rule{0.5\linewidth}{0.5pt}\end{center}

\begin{enumerate}
\def\labelenumi{\arabic{enumi}.}
\setcounter{enumi}{7}
\tightlist
\item
  Using the sample redcap dataset available at
  {[}\url{https://jlucasmckay.bmi.emory.edu/global/bmi510/gait.csv}{]},
  identify the unique patients and summarize the ratio of women to men.
  Also report how many missing values there are. (The sex variable codes
  0 for male, 1 for female.) \textbf{(1/2 point)} Then, calculate (and
  plot) the (Pearson's) correlation between gait speed and cadence.
  \textbf{(1/2 point)}
\end{enumerate}

\begin{Shaded}
\begin{Highlighting}[]

\FunctionTok{library}\NormalTok{(dplyr)}
\FunctionTok{library}\NormalTok{(ggcorrplot)}
\FunctionTok{library}\NormalTok{(corrplot)}
\FunctionTok{library}\NormalTok{(ggcorrplot)}



\NormalTok{redcap }\OtherTok{\textless{}{-}} \FunctionTok{read.csv}\NormalTok{(}\StringTok{"https://jlucasmckay.bmi.emory.edu/global/bmi510/gait.csv"}\NormalTok{)}
\NormalTok{unique\_patients }\OtherTok{\textless{}{-}} \FunctionTok{unique}\NormalTok{(redcap}\SpecialCharTok{$}\NormalTok{record\_id)}

\FunctionTok{cat}\NormalTok{(}\StringTok{"No of Unique patients are : "}\NormalTok{,}\FunctionTok{length}\NormalTok{(unique\_patients))}
\DocumentationTok{\#\# No of Unique patients are :  935}

\CommentTok{\#we need to filter out the rows where the reocrd\_id reappears before calculating the ratio of men to women}
\NormalTok{redcap\_unique }\OtherTok{\textless{}{-}}\NormalTok{ redcap }\SpecialCharTok{\%\textgreater{}\%} \FunctionTok{filter}\NormalTok{(}\SpecialCharTok{!}\FunctionTok{duplicated}\NormalTok{(record\_id))}

\NormalTok{m }\OtherTok{\textless{}{-}}\NormalTok{ redcap\_unique}\SpecialCharTok{$}\NormalTok{sex }\SpecialCharTok{==} \DecValTok{0}
\NormalTok{f }\OtherTok{\textless{}{-}}\NormalTok{ redcap\_unique}\SpecialCharTok{$}\NormalTok{sex }\SpecialCharTok{==} \DecValTok{1}

\NormalTok{men }\OtherTok{\textless{}{-}} \FunctionTok{length}\NormalTok{(m[m }\SpecialCharTok{==} \ConstantTok{TRUE}\NormalTok{])}
\NormalTok{women }\OtherTok{\textless{}{-}} \FunctionTok{length}\NormalTok{(f[f }\SpecialCharTok{==} \ConstantTok{TRUE}\NormalTok{])}

\NormalTok{ratio\_of\_men\_women }\OtherTok{\textless{}{-}}\NormalTok{ men}\SpecialCharTok{/}\NormalTok{women}

\FunctionTok{cat}\NormalTok{(}\StringTok{"}\SpecialCharTok{\textbackslash{}n}\StringTok{Ratio of men to women :"}\NormalTok{,ratio\_of\_men\_women)}
\DocumentationTok{\#\# }
\DocumentationTok{\#\# Ratio of men to women : 1.301811}

\NormalTok{missing\_vals }\OtherTok{\textless{}{-}} \FunctionTok{is.na}\NormalTok{(redcap\_unique}\SpecialCharTok{$}\NormalTok{sex)}
\NormalTok{missing\_values }\OtherTok{\textless{}{-}} \FunctionTok{length}\NormalTok{(missing\_vals[missing\_vals }\SpecialCharTok{==} \ConstantTok{TRUE}\NormalTok{])}

\FunctionTok{cat}\NormalTok{(}\StringTok{"}\SpecialCharTok{\textbackslash{}n}\StringTok{The sex column has "}\NormalTok{,missing\_values,}\StringTok{" missing values}\SpecialCharTok{\textbackslash{}n}\StringTok{"}\NormalTok{)}
\DocumentationTok{\#\# }
\DocumentationTok{\#\# The sex column has  209  missing values}

\NormalTok{redcap\_without\_na }\OtherTok{\textless{}{-}}\NormalTok{ redcap }\SpecialCharTok{\%\textgreater{}\%} \FunctionTok{filter}\NormalTok{(}\SpecialCharTok{!}\FunctionTok{is.na}\NormalTok{(speed) }\SpecialCharTok{\&} \SpecialCharTok{!}\FunctionTok{is.na}\NormalTok{(cadence))}

\NormalTok{df\_x }\OtherTok{\textless{}{-}} \FunctionTok{data.frame}\NormalTok{(redcap\_without\_na}\SpecialCharTok{$}\NormalTok{speed)}

\NormalTok{df\_y }\OtherTok{\textless{}{-}} \FunctionTok{data.frame}\NormalTok{(redcap\_without\_na}\SpecialCharTok{$}\NormalTok{cadence)}


\NormalTok{correlation }\OtherTok{\textless{}{-}} \FunctionTok{cor}\NormalTok{(df\_x, df\_y)}
\NormalTok{correlation\_matrix }\OtherTok{\textless{}{-}} \FunctionTok{cor}\NormalTok{(}\FunctionTok{cbind}\NormalTok{(df\_x, df\_y))}

\FunctionTok{ggcorrplot}\NormalTok{(correlation\_matrix, }\AttributeTok{method =} \StringTok{"circle"}\NormalTok{)}
\end{Highlighting}
\end{Shaded}

\includegraphics{HW-04_files/figure-latex/unnamed-chunk-8-1.pdf}

\begin{Shaded}
\begin{Highlighting}[]

\CommentTok{\#df\_x \textless{}{-} data.frame(x\_l \textless{}{-} redcap\_without\_na$speed, lab \textless{}{-} rep("gait speed", nrow(df\_x)))}

\CommentTok{\#df\_y \textless{}{-} data.frame(x\_l \textless{}{-} redcap\_without\_na$cadence, lab \textless{}{-} rep("cadence", nrow(df\_y)))}
\CommentTok{\#colnames(df\_x) \textless{}{-} c("x\_l", "lab")}
\CommentTok{\#colnames(df\_y) \textless{}{-} c("x\_l", "lab")}

\CommentTok{\#redcap\_without\_na\_combined \textless{}{-} bind\_rows(df\_x,df\_y)}


\FunctionTok{ggplot}\NormalTok{(redcap\_without\_na, }\FunctionTok{aes}\NormalTok{(}\AttributeTok{x =}\NormalTok{ speed, }\AttributeTok{y =}\NormalTok{ cadence)) }\SpecialCharTok{+}
\FunctionTok{geom\_point}\NormalTok{() }\SpecialCharTok{+}
\FunctionTok{geom\_smooth}\NormalTok{(}\AttributeTok{method=}\StringTok{"lm"}\NormalTok{, }\AttributeTok{se=}\ConstantTok{FALSE}\NormalTok{)}\SpecialCharTok{+}
\FunctionTok{ggtitle}\NormalTok{(}\FunctionTok{paste}\NormalTok{(}\StringTok{"Pearson\textquotesingle{}s correlation coefficient: "}\NormalTok{, }\FunctionTok{round}\NormalTok{(correlation, }\DecValTok{2}\NormalTok{))) }\SpecialCharTok{+}
\FunctionTok{xlab}\NormalTok{(}\StringTok{"Gait speed"}\NormalTok{) }\SpecialCharTok{+}
\FunctionTok{ylab}\NormalTok{(}\StringTok{"Cadence"}\NormalTok{)}
\end{Highlighting}
\end{Shaded}

\includegraphics{HW-04_files/figure-latex/unnamed-chunk-8-2.pdf}

\begin{center}\rule{0.5\linewidth}{0.5pt}\end{center}

\begin{enumerate}
\def\labelenumi{\arabic{enumi}.}
\setcounter{enumi}{8}
\tightlist
\item
  We have not discussed (at least at length) joint frequency functions /
  probability mass functions, but they are a generalization of
  probability mass functions for a single random variable. Say the joint
  frequency function of two discrete random variables, X and Y, is as
  follows:
\end{enumerate}

\begin{longtable}[]{@{}lllll@{}}
\toprule()
X & Y=1 & Y=2 & Y=3 & Y=4 \\
\midrule()
\endhead
1 & 0.10 & 0.05 & 0.02 & 0.02 \\
2 & 0.05 & 0.20 & 0.05 & 0.02 \\
3 & 0.02 & 0.05 & 0.20 & 0.04 \\
4 & 0.02 & 0.02 & 0.04 & 0.10 \\
\bottomrule()
\end{longtable}

These data are available at
{[}\url{https://jlucasmckay.bmi.emory.edu/global/bmi510/joint_frequency.csv}{]}
Columns 2 through 4 are named Y1, Y2, Y3, and Y4. Completely melt the
data \textbf{(1/2 point)} and extract numerical values from the codes
\texttt{Y1}, \texttt{Y2}, etc. \textbf{(1/2 point)}

\begin{Shaded}
\begin{Highlighting}[]

\FunctionTok{library}\NormalTok{(tidyr)}
\FunctionTok{library}\NormalTok{(reshape2)}


\NormalTok{data }\OtherTok{\textless{}{-}} \FunctionTok{read.csv}\NormalTok{(}\StringTok{"https://jlucasmckay.bmi.emory.edu/global/bmi510/joint\_frequency.csv"}\NormalTok{)}
\NormalTok{melted\_data }\OtherTok{\textless{}{-}} \FunctionTok{melt}\NormalTok{(data, }\AttributeTok{id.vars=}\StringTok{"X"}\NormalTok{)}

\FunctionTok{print}\NormalTok{(melted\_data}\SpecialCharTok{$}\NormalTok{value)}
\DocumentationTok{\#\#  [1] 0.10 0.05 0.02 0.02 0.05 0.20 0.05 0.02 0.02 0.05 0.20 0.04 0.02 0.02 0.04}
\DocumentationTok{\#\# [16] 0.10}
\end{Highlighting}
\end{Shaded}

\begin{center}\rule{0.5\linewidth}{0.5pt}\end{center}

\begin{enumerate}
\def\labelenumi{\arabic{enumi}.}
\setcounter{enumi}{9}
\tightlist
\item
  Find and plot the marginal frequency functions of X and Y. In the
  two-by-two table above, these would be the row and column sums. (The
  margins.) To what do the frequencies sum? \textbf{(1 point)}
\end{enumerate}

\begin{Shaded}
\begin{Highlighting}[]


\NormalTok{marginal\_frequency\_X }\OtherTok{\textless{}{-}} \FunctionTok{c}\NormalTok{(}\FloatTok{0.10} \SpecialCharTok{+} \FloatTok{0.05} \SpecialCharTok{+} \FloatTok{0.02} \SpecialCharTok{+} \FloatTok{0.02}\NormalTok{,}
                          \FloatTok{0.05} \SpecialCharTok{+} \FloatTok{0.20} \SpecialCharTok{+} \FloatTok{0.05} \SpecialCharTok{+} \FloatTok{0.02}\NormalTok{,}
                          \FloatTok{0.02} \SpecialCharTok{+} \FloatTok{0.05} \SpecialCharTok{+} \FloatTok{0.20} \SpecialCharTok{+} \FloatTok{0.04}\NormalTok{,}
                          \FloatTok{0.02} \SpecialCharTok{+} \FloatTok{0.02} \SpecialCharTok{+} \FloatTok{0.04} \SpecialCharTok{+} \FloatTok{0.10}\NormalTok{)}

\NormalTok{marginal\_frequency\_Y }\OtherTok{\textless{}{-}} \FunctionTok{c}\NormalTok{(}\FloatTok{0.10} \SpecialCharTok{+} \FloatTok{0.05} \SpecialCharTok{+} \FloatTok{0.02} \SpecialCharTok{+} \FloatTok{0.02}\NormalTok{,}
                          \FloatTok{0.05} \SpecialCharTok{+} \FloatTok{0.20} \SpecialCharTok{+} \FloatTok{0.05} \SpecialCharTok{+} \FloatTok{0.02}\NormalTok{,}
                          \FloatTok{0.02} \SpecialCharTok{+} \FloatTok{0.05} \SpecialCharTok{+} \FloatTok{0.20} \SpecialCharTok{+} \FloatTok{0.04}\NormalTok{,}
                          \FloatTok{0.02} \SpecialCharTok{+} \FloatTok{0.02} \SpecialCharTok{+} \FloatTok{0.04} \SpecialCharTok{+} \FloatTok{0.10}\NormalTok{)}

\FunctionTok{cat}\NormalTok{(}\StringTok{"Marg freq x:"}\NormalTok{, marginal\_frequency\_X, }\StringTok{"}\SpecialCharTok{\textbackslash{}n}\StringTok{Marg freq y"}\NormalTok{,marginal\_frequency\_Y)}
\DocumentationTok{\#\# Marg freq x: 0.19 0.32 0.31 0.18 }
\DocumentationTok{\#\# Marg freq y 0.19 0.32 0.31 0.18}

\FunctionTok{barplot}\NormalTok{(marginal\_frequency\_X, }\AttributeTok{names.arg=}\FunctionTok{c}\NormalTok{(}\DecValTok{1}\NormalTok{,}\DecValTok{2}\NormalTok{,}\DecValTok{3}\NormalTok{,}\DecValTok{4}\NormalTok{), }\AttributeTok{xlab=}\StringTok{"X"}\NormalTok{, }\AttributeTok{ylab=}\StringTok{"Frequency"}\NormalTok{, }\AttributeTok{main =} \StringTok{"marginal freq x"}\NormalTok{)}
\end{Highlighting}
\end{Shaded}

\includegraphics{HW-04_files/figure-latex/unnamed-chunk-10-1.pdf}

\begin{Shaded}
\begin{Highlighting}[]
\FunctionTok{barplot}\NormalTok{(marginal\_frequency\_Y, }\AttributeTok{names.arg=}\FunctionTok{c}\NormalTok{(}\DecValTok{1}\NormalTok{,}\DecValTok{2}\NormalTok{,}\DecValTok{3}\NormalTok{,}\DecValTok{4}\NormalTok{), }\AttributeTok{xlab=}\StringTok{"Y"}\NormalTok{, }\AttributeTok{ylab=}\StringTok{"Frequency"}\NormalTok{, }\AttributeTok{main =} \StringTok{"marginal freq y"}\NormalTok{)}
\end{Highlighting}
\end{Shaded}

\includegraphics{HW-04_files/figure-latex/unnamed-chunk-10-2.pdf}

\begin{Shaded}
\begin{Highlighting}[]

\CommentTok{\#The frequencies represent the distribution of the discrete random variables X and Y. The sum of the marginal frequencies represent the total count of occurrences of X and Y. }
\end{Highlighting}
\end{Shaded}


\end{document}
